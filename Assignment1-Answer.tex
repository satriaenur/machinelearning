\documentclass{article}
\usepackage{amsmath}

\begin{document}
Nama : Satria Nur Hidayatullah

NIM : 1103130035

Kelas : MachineLearning-Gab-01\\



\textbf{Section 1 : Theory}
\begin{enumerate}
	\item Machine learning adalah sebuah metode dalam pembuatan suatu task dalam komputer, yang bertujuan membuat
	mesin menirukan manusia dalam hal pembelajaran. Sehingga pada tujuan akhirnya terbentuk suatu program yang dapat
	memahami permasalahan yang diberikan serta memberikan solusi yang diinginkan tanpa harus mendapat perintah dari manusia.
	\item Contohnya adalah suatu program yang dapat mengenali gambar asli atau gambar hasil editan yang biasa disebut dengan \textit{image spoofing}
	\item
	\begin{enumerate}
		\item untuk perhitungan cosine hanya digunakan range [0 - 1] sedangkan cosine correlation memiliki nilai rentang [-1 - 1\
		\item Ya, cosine maksimal ndengan nilai 1 menunjukkan bahwa 2 objek tersebut adalah objek yang "similar" apabila digambarkan dengan
		vektor arah, maka 2 objek yang memiliki nilai cosine sama dengan 1 adalah 2 objek yang memiliki arah yang sama, sedangkan 2 objek
		yang memiliki nilai cosine 0, artinya orthogonal atau arah ke-2 objek tersebut saling tegak lurus
		\item Berdasarkan fungsi:\\
		\begin{equation}\label{cossim}
			CosSim(x,y) = \dfrac{\sum_{i}^{n}x_{i}y_{i}}{\sqrt{\sum_{i}^{n}x_{i}^{2}}\sqrt{\sum_{i}^{n}y_{i}^{2}}}
		\end{equation}
		\begin{equation}\label{corr}
			Corr(x,y) = \dfrac{\sum_{i}^{n}(x_{i} - \bar{x})(y_{i} - \bar{y)}}{\sqrt{\sum_{i}^{n}(x_{i} - \bar{x})^{2}}\sqrt{\sum_{i}^{n}(y_{i} - \bar{y})^{2}}}
		\end{equation}
		sehingga diperoleh fungsi:\\
		\begin{equation}\label{corrcos}
			Corr(x,y) = CosSim(x - \bar{x}, y - \bar{y})
		\end{equation}
		sehingga dapat dibuktikan dari fungsi \eqref{corrcos} bahwa correlation didapatkan dari hasil cosine similarity yang telah di kurangi oleh rata-rata. Sehingga correlation adalah fungsi yang digunakan untuk mencari cosine similarity dengan mengambil nilai asli dari vector itu sendiri, karena pengurangan terhadap rata-rata digunakan untuk mencari nilai asli dari suatu data.
		\item 
	\end{enumerate}
\end{enumerate}
\end{document}