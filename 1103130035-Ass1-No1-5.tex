\documentclass{article}
\usepackage{amsmath}

\begin{document}
Nama : Satria Nur Hidayatullah

NIM : 1103130035

Kelas : MachineLearning-Gab-01\\



\textbf{Section 1 : Theory}
\begin{enumerate}
	\item Machine learning adalah sebuah metode dalam pembuatan suatu task dalam komputer, yang bertujuan membuat
	mesin menirukan manusia dalam hal pembelajaran. Sehingga pada tujuan akhirnya terbentuk suatu program yang dapat
	memahami permasalahan yang diberikan serta memberikan solusi yang diinginkan tanpa harus mendapat perintah dari manusia.
	\item Contohnya adalah suatu program yang dapat mengenali gambar asli atau gambar hasil editan yang biasa disebut dengan \textit{image spoofing}
	\item
	\begin{enumerate}
		\item untuk perhitungan cosine hanya digunakan range [0 - 1] sedangkan cosine correlation memiliki nilai rentang [-1 - 1\
		\item Tidak tentu, cosine maksimal dengan nilai 1 menunjukkan bahwa 2 objek tersebut adalah objek yang "similar" apabila digambarkan dengan
		vektor arah, maka 2 objek yang memiliki nilai cosine sama dengan 1 adalah 2 objek yang memiliki arah yang sama. Arah yang sama artinya adalah atribut yang dimiliki dibedakan dengan nilai yang konstan.
		\item Berdasarkan fungsi:\\
		\begin{equation}\label{cossim}
			CosSim(x,y) = \dfrac{\sum_{i}^{n}x_{i}y_{i}}{\sqrt{\sum_{i}^{n}x_{i}^{2}}\sqrt{\sum_{i}^{n}y_{i}^{2}}}
		\end{equation}
		\begin{equation}\label{corr}
			Corr(x,y) = \dfrac{\sum_{i}^{n}(x_{i} - \bar{x})(y_{i} - \bar{y)}}{\sqrt{\sum_{i}^{n}(x_{i} - \bar{x})^{2}}\sqrt{\sum_{i}^{n}(y_{i} - \bar{y})^{2}}}
		\end{equation}
		sehingga diperoleh fungsi:\\
		\begin{equation}\label{corrcos}
			Corr(x,y) = CosSim(x - \bar{x}, y - \bar{y})
		\end{equation}
		sehingga dapat dibuktikan dari fungsi \eqref{corrcos} bahwa correlation didapatkan dari hasil cosine similarity yang telah di kurangi oleh rata-rata. Sehingga correlation adalah fungsi yang digunakan untuk mencari cosine similarity dengan mengambil nilai asli dari vector itu sendiri, karena pengurangan terhadap rata-rata digunakan untuk mencari nilai asli dari suatu data. Dan apabila nilai rata rata dari x dan y = 0, maka $ corr(x,y) = cos(x,y) $
		\item misal $ x $ dan $ y $ adalah 2 vektor yang memiliki panjan $ L_{2} $ = 1. Untuk kasus vektor seperti ini, nilai variansi dari ke-2 vektor adalah $ n $ kali jumlah akar dari nilai atribut dan nilai korelasi nya adalah hasil dari dot product kedua vektor dibagi dengan $ n $.\\
		\begin{align*}
			d(x,y)
				&= \sqrt{\sum_{k = 1}^{n}(x_{k} - y_{k})^{2}} \\
				&= \sqrt{\sum_{k = 1}^{n}x_{k}^{2} - 2x_{k}y_{k} + y_{k}^{2}} \\
				&= \sqrt{1 - 2\cos(x,y) + 1} \\
				&= \sqrt{2(1-\cos(x,y))}
		\end{align*}
		\item misal $ x $ dan $ y $ adalah 2 vektor yang memiliki nilai mean 0 dan standar deviasi 1. Untuk kasus vektor seperti ini, nilai variansi dari ke-2 vektor adalah $ n $ kali jumlah akar dari nilai atribut dan nilai korelasi nya adalah hasil dari dot product kedua vektor dibagi dengan $ n $.\\
		\begin{align*}
		d(x,y)
		&= \sqrt{\sum_{k = 1}^{n}(x_{k} - y_{k})^{2}} \\
		&= \sqrt{\sum_{k = 1}^{n}x_{k}^{2} - 2x_{k}y_{k} + y_{k}^{2}} \\
		&= \sqrt{n - 2ncorr(x,y) + n} \\
		&= \sqrt{2n(1-corr(x,y))}
		\end{align*}
	\end{enumerate}
	\item
	\begin{enumerate}
		\item Yang pertama dengan cara mengitung jarak dengan suatu centroid, dengan menggunakan nilai euclidian distance, seperti saat melakukan clustering. Yang kedua adalah dengan menggunakan kedekatan antar 2 data. Salah satunya dengan menggunakan nilai minimum similarity, atau maksimum similarity.
		\item Salah satu caranya adalah menghitung jarak antara centroid dengan 2 titik nilai.
		\item Dengan menghitung jarak dari kedua nilai tersebut dengan 1 titik centorid. Atau dengan cara mengambil langsung nilai kedekatan minimum atau maksimum.
	\end{enumerate}
	\item 
	\begin{enumerate}
		\item Hasil log pada perhitungan invers menjadi $ n $ jumlah dokumen, sehingga memperoleh nilai maksimal dari frekuensi dokumen invers.
		\item Hasil log pada perhitungan invers menjadi 0, sehingga memiliki nilai minimum frekuensi dokumen invers.
		\item Hasil invers menunjukkan bahwa kata yang muncul di semua dokumen tidak bisa membedakan ciri-ciri antar dokumen, sedangkan kata yang hanya di sedikit dokumen, bisa membedakan antar dokumen.
		\item Salah satu penggunaannya adalah ketika digunakan pada perhitungan suatu film yang disukai oleh beberapa orang dengan kesenangan genre tersendiri. Bisa digunakan untuk film x yang sangat disukai oleh orang-orang dengan kesenangan genre x ketika  dia hanya disukai oleh sekelompok orang.
	\end{enumerate}
\end{enumerate}
Reference: Hampir semua jawaban saya dapatkan rata-rata dari mencoba memahami buku Introduction to Data Mining karya Pang-Ning Tan, Michael Steinbach, Vipin Kumar. Namun ada beberapa yang kata-katanya hampir mirip karena tidak bisa memahami. Dan buku ini saya dapatkan dengan cara googling soal secara langsung pada saat menyerah.hehe
\end{document}